\documentclass[11pt]{article}
\usepackage{fancyhdr}
\usepackage[vmargin=3.5cm, hmargin=2cm]{geometry}
\usepackage{amsmath, amsfonts, amsthm}
\usepackage{graphicx}
\usepackage{moreverb}
\usepackage{enumerate}
\usepackage{bm}
\usepackage[tiny,compact]{titlesec}
\pagestyle{fancy}
\headheight 14pt
\lhead{UC Berkeley}
\chead{}
\rhead{Stat 88 Fall 2016}
\rfoot{}
\cfoot{\thepage}
\lfoot{}

\parindent0pt  % to stop indenting paragraphs
\parskip1.5ex  % to insert vertical space between paragraphs

% Different font in captions
\newcommand{\captionfonts}{\small}

\makeatletter  % Allow the use of @ in command names
\long\def\@makecaption#1#2{%
  \vskip\abovecaptionskip
  \sbox\@tempboxa{{\captionfonts #1: #2}}%
  \ifdim \wd\@tempboxa >\hsize
    {\captionfonts #1: #2\par}
  \else
    \hbox to\hsize{\hfil\box\@tempboxa\hfil}%
  \fi
  \vskip\belowcaptionskip}
\makeatother   % Cancel the effect of \makeatletter

\newcommand{\V}{\mathrm{Var}}
\newcommand{\E}{\mathbb{E}}
\newcommand{\mbf}{\mathbf}
\newcommand{\mr}{\mathrm}
\newcommand{\yiobs}{Y_i^\mr{obs}}
\newcommand{\yobs}{Y^\mr{obs}}
\newcommand{\yimis}{Y_i^\mr{mis}}
\newcommand{\ymis}{Y^\mr{mis}}
\newcommand{\N}{\mathcal{N}}
\newcommand{\utilde}{\underset{\widetilde{}}}
\newcommand{\txt}{\texttt}

\begin{document}
\centerline{\textbf{Homework 5}}
\centerline{Due on Gradescope 10/25/2016 at 4:00PM (Before Lecture)}

The first two problems should look familiar!

\begin{enumerate}
\item \underline{Deriving Parameters, to appear in TMD}:
The goal of this problem is for you to use the definitions and notation from class in familiar
contexts. For each random variable, derive the expectation $E(X)$, the variance $\V(X)$, and the standard deviation $SD(X)$ from their definitions (i.e., work out the sum). 
\begin{enumerate}
\item $X$ is the number of spots that show on one roll of a fair six-sided die.
\item $X$ is an ``indicator'' random variable; it has the value 1 with probability p, and the value 0
with probability 1 − p. This random variable is a Boolean, that is, it can only be 0 or 1. Just as
0’s and 1’s are powerful in computing, so also indicators are powerful in probability theory. You’ll see how next week.
\item $X$ is the number of heads in one toss of a fair coin.
\item $X$ is the number of heads in two tosses of a fair coin.
\item $X$ is the number of red cards among two cards picked at random without replacement from
a standard deck (52 cards of which 26 are red).
\end{enumerate}

\item \underline{Stock Price}:
    ({\bf Note:} In this problem, you may use these facts presented in class: If $X$ is a binomial random variable with size $n$ and probability $p$, $E(X) = np$ and $\V(X) = np(1-p)$.)

Suppose that TechCo is one of San Francisco's hottest publicly traded tech startups, and its stock price moves in the following way: every day, it either increases by \$1 with probability $p$ or decreases by \$1 with probability $1-p$, and the change on each day is independent. Let $Z$ be the change in the price of TechCo's stock over two weeks; that is, $Z$ is the price of TechCo's stock on Oct 19 minus the price of TechCo's stock today, Oct 5.

\begin{enumerate}
    \item What is $E(Z)$?
    \item What is $SD(Z)$?
    \item Suppose that $p$ is 0.51, so the stock price has a very slight upward drift. How many days would it take for $E(Z)$ to be more than 2 standard deviations ($SD(X)$) away from 0?
\end{enumerate}

\item \underline{Polling Margin of Error}:
    ({\bf Note:} In this problem, you may use these facts presented in class: If $X$ is a binomial random variable with size $n$ and probability $p$, $E(X) = np$ and $\V(X) = np(1-p)$.)

    When pollsters conduct surveys of political opinions, they draw a sample of size $n$ without replacement from a population of size $N$, where $N$ is much larger than $n$. Nonetheless, they often model the number of people supporting a particular candidate (for the purposes of this quesion, let's say Clinton), $X$, as a binomial random variable with size $n$ and probability $p$. In this case, the pollster does not know the true value $p$, but we can write down expectations and variances of important quantities in terms of $p$.

    \begin{enumerate}
        \item Technically, what are the requirements for a random variable $X$ to be binomially distributed? Which of these requirements are violated in this case? Intuitively, why is the binomial model approximately correct here? (Hint: Sizes of $N$ and $n$ matter).
        \item Define $Y$ as the \emph{proportion} of respondents who say they will vote for Clinton, $\frac{X}{n}$. What is $E(Y)$?
        \item Using the same definition of $Y$ as above, what is $\V(Y)$?
        \item For what value of $p$ is $\V(Y)$ maximized? Use calculus or an algebraic argument.
        \item Simple polls will report $Y$ as their estimate of the unknown proportion of Clinton supporters $p$. They will also report a ``margin of error'', which is 2 times the worst case standard deviation of $Y$ (that is, the largest SD among all potential parameters $p$). For a poll with $n$ respondents, what is the margin of error?
    \end{enumerate}

\end{enumerate}
\end{document}
